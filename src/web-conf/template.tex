%%%%%%%%%%%%%%%%%%%%%%% file template.tex %%%%%%%%%%%%%%%%%%%%%%%%%
%
% This is a template file for Web of Conferences Journal
%
% Copy it to a new file with a new name and use it as the basis
% for your article
%
%%%%%%%%%%%%%%%%%%%%%%%%%% EDP Science %%%%%%%%%%%%%%%%%%%%%%%%%%%%
%
%%%\documentclass[option comma separated list]{webofc}
%%%Three important options:
%%% "epj" for EPJ Web of Conferences Journal
%%% "bio" for BIO Web of Conferences Journal
%%% "mat" for MATEC Web of Conferences Journal
%%% "itm" for ITM Web of Conferences Journal
%%% "e3s" for E3S Web of Conferences Journal
%%% "shs" for SHS Web of Conferences Journal
%%% "twocolumn" for typesetting an article in two columns format (default one column)
\documentclass{webofc}
\usepackage[varg]{txfonts}   % Web of Conferences font
%
% Put here some packages required or/and some personnal commands
%
% Important: please activate and fill the "wocname" command with the exact title of the series for conferences not included in any of the series listed on the top
%
%\wocname{?????????}
%
% Very important: please fill the "woctitle" command with the exact title of the conference
%
\woctitle{?????????}
%
%
\begin{document}
%
\title{Insert your title here}
%
% subtitle is optionnal
%
%%%\subtitle{Do you have a subtitle?\\ If so, write it here}

\author{First author\inst{1,3}\fnsep\thanks{\email{Mail address for first
    author}} \and
        Second author\inst{2}\fnsep\thanks{\email{Mail address for second
             author if necessary}} \and
        Third author\inst{3}\fnsep\thanks{\email{Mail address for last
             author if necessary}}
        % etc.
}

\institute{Insert the first address here 
\and
           the second here 
\and
           Last address
          }

\abstract{%
  Insert your english abstract here.
}
%
\maketitle
%
\section{Introduction}
\label{intro}
Your text comes here. Separate text sections with
\section{Section title}
\label{sec-1}
For bibliography use \cite{RefJ}
\subsection{Subsection title}
\label{sec-2}
Don't forget to give each section, subsection, subsubsection, and
paragraph a unique label (see Sect.~\ref{sec-1}).

For one-column wide figures use syntax of figure~\ref{fig-1}
\begin{figure}
% Use the relevant command for your figure-insertion program
% to insert the figure file.
\centering
\includegraphics[width=1cm,clip]{tiger}
\caption{Please write your figure caption here}
\label{fig-1}       % Give a unique label
\end{figure}

For two-column wide figures use syntax of figure~\ref{fig-2}
\begin{figure*}
\centering
% Use the relevant command for your figure-insertion program
% to insert the figure file. See example above.
% If not, use
\vspace*{5cm}       % Give the correct figure height in cm
\caption{Please write your figure caption here}
\label{fig-2}       % Give a unique label
\end{figure*}

For figure with sidecaption legend use syntax of figure
\begin{figure}
% Use the relevant command for your figure-insertion program
% to insert the figure file.
\centering
\sidecaption
\includegraphics[width=5cm,clip]{tiger}
\caption{Please write your figure caption here}
\label{fig-3}       % Give a unique label
\end{figure}

For tables use syntax in table~\ref{tab-1}.
\begin{table}
\centering
\caption{Please write your table caption here}
\label{tab-1}       % Give a unique label
% For LaTeX tables you can use
\begin{tabular}{lll}
\hline
first & second & third  \\\hline
number & number & number \\
number & number & number \\\hline
\end{tabular}
% Or use
\vspace*{5cm}  % with the correct table height
\end{table}
%
% BibTeX or Biber users please use (the style is already called in the class, ensure that the "woc.bst" style is in your local directory)
% \bibliography{name or your bibliography database}
%
% Non-BibTeX users please use
%
\begin{thebibliography}{}
%
% and use \bibitem to create references.
%
\bibitem{RefJ}
% Format for Journal Reference
Journal Author, Journal \textbf{Volume}, page numbers (year)
% Format for books
\bibitem{RefB}
Book Author, \textit{Book title} (Publisher, place, year) page numbers
% etc
\end{thebibliography}

\end{document}

% end of file template.tex

