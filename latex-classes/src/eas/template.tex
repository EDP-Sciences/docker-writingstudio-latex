%%%%%%%%%%%%%%%%%%%%%%% file template.tex %%%%%%%%%%%%%%%%%%%%%%%%%
%
% This is a template file for these proceedings 
%
% Copy it to a new file with a new name and use it as the basis
% for your article
%
%%%%%%%%%%%%%%%%%%%%%%%%   EDP Sciences  %%%%%%%%%%%%%%%%%%%%%%%%%%
%
\documentclass{eas}
\usepackage{graphicx}
%
%%%%%%%%%%%%%--PREAMBLE--%%%%%%%%%%%%%%%%%%
%%-----------------------------
%         ...........
%         your macros
%         ...........
%%-------------------------%%----
%%%%%%%%%%%%%%%--BODY--%%%%%%%%%%%%%%%%%%
%
%\TitreGlobal{The Title of this Volume}
%
\begin{document}

%%-----------------------------
%%      the top matter
%%-----------------------------
\title{...} 
%
\author{...}\address{...}
\author{...}\address{...}
\author{...}\address{...}
%
%
\begin{abstract}
...
\end{abstract}
%
\maketitle
%%-----------------------------
%%      your text
%%-----------------------------
\section{Introduction}
The InfraRed Astronomy Satellite (IRAS; Beichman {\em et al.\/} \cite{Bei}) 
ushered in the era of space-based infrared astronomy in a dramatic fashion, 
revealing a stunningly rich infrared sky, unanticipated from the bits of 
infrared data previously and heroically collected from the ground...
IRAS had a profound influence on
astronomy in general, not just the infrared, because it represented such a
very large gain in sensitivity and spatial coverage, comparable perhaps to
going from attempting visual astronomy in daylight to observing in a dark
night (Beichman \cite{ref1987}; Soifer \etal \cite{so1987}).  Another
significant factor in this influence was the  dissemination
of data products from IRAS, including source catalogs, image atlases and sky
brightness estimates, generated with great care and well characterized and
documented as to reliability, completeness and other statistical
attributes.
\section{...}
\subsection{...}
\subsection{...}
...
%%-----------------------------
%%      your bibliography
%%-----------------------------
\begin{thebibliography}{99}
\bibitem[1994]{alref1} Aalto, S. \etal\  1994, A\&A, 286, 365.
%%% Using \cite{Bei} in the text
\bibitem[1986]{Bei} Beichman, C.A., Neugebauer, G., Habing,
   H., Clegg, P.E. \& Chester, T.C. 1988, editors, {\it ``IRAS Catalogs and
   Atlases: Explanatory Supplement''}, NASA RP-1190 (Washington: NASA)
\bibitem[1987]{ref1987} Beichman, C.A. 1987, ARA\&A, 25, 521
\bibitem[1987]{so1987} Soifer, B.T., Houck, J.R. and Neugebauer, G. 1987, ARAA, 25, 187
\end{thebibliography}
\end{document}

